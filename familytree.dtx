% \iffalse
% Copyright 2022 Jiro Senju
%
% This package is free software; you can redistribute it and/or modify
% it under the terms of the GNU General Public License as published by
% the Free Software Foundation; either version 2 of the License, or
% any later version.
%
% This package is distributed in the hope that it will be useful,
% but WITHOUT ANY WARRANTY; without even the implied warranty of
% MERCHANTABILITY or FITNESS FOR A PARTICULAR PURPOSE.  See the
% GNU General Public License for more details.
%
% You should have received a copy of the GNU General Public License
% along with this package.  If not, see <http://www.gnu.org/licenses/>.
%
%<package>\NeedsTeXFormat{LaTeX2e}
%<package>\ProvidesPackage{familytree}%
%<package>[2022/06/18 v2.0 familytree]
%
%<*driver>
\documentclass[draft]{ltxdoc}
\let\Ocmd=\cmd
\renewcommand{\cmd}[1]{%
  %\hskip-\parindent
  \parbox{.95\textwidth}{\Ocmd{#1}}%
}
%
\usepackage[scaled=.95]{erewhon}
\usepackage[erewhon,vvarbb,bigdelims]{newtxmath}
\usepackage[scale=.95]{Chivo}

%
\usepackage[dvipdfmx,final]{graphicx}
\usepackage[final]{listings}
\lstset{basicstyle=\small\ttfamily,
  language=TeX,
  columns=[c]flexible,
  tabsize=4,
  %frame=tb,
  keepspaces=true,
  %lineskip=-.1em,
  belowskip=\medskipamount}
\usepackage{multicol}
\usepackage{needspace}
%
% hyperref should come last
\usepackage[dvipdfmx,final,pdfusetitle]{hyperref}
% bookmarksdepth=section
% pdffitwindow=true
% pdfpagetransition=Dissolve
% pdfstartview=FitB
\hypersetup{
  hyperfootnotes=false,
  colorlinks=true,
  linkcolor=blue,
  % anchorcolor=black,
  % citecolor=black,
  % urlcolor=black,
  bookmarks=true,
  bookmarksnumbered=true,
  setpagesize=false,
  %pdftitle={},
  %pdfauthor={},
  pdfpagelayout=SinglePage,
  pdfpagemode=UseOutlines,
  pdfstartview=FitH
}
%
\usepackage{\jobname}
%
\makeatletter
\def\meta@font@select{\slshape}
\def\fps@table{htbp}
\let\part=\@gobble
%
\newcount\ft@lastsection
\newcommand{\subsectImpl}[1][\subsection]{%
  \iftrue% debugging code
    \@tempcnta=\numexpr\thesection + 1\relax%
    \ifnum\@tempcnta>\ft@lastsection\else%
      Jump to \hyperlink{section.\the\@tempcnta}{next section}%
    \fi%
  \fi%
  \needspace{4\baselineskip}%
  #1{Implementation}%
}
\makeatother
%
\newcommand{\parag}[1]{%
  \allowbreak%
  \paragraph{#1}\nopagebreak\hskip0pt\nopagebreak%
  %\medskip%
}
\newcounter{CS}[section]
\newcommand{\CS}{%
  \ifnum\theCS=0%
    control sequence%
    \stepcounter{CS}%
  \else%
    CS%
  \fi%
}
\newcommand{\srcfig}[2][]{%
  \lstinputlisting{\Dir/#2print}%
  \nopagebreak[4]%
  \hfil\includegraphics[#1]{\Dir/#2.pdf}%
}
\newcommand{\NoDescription}{\vspace*{-.9\baselineskip}}
\newcommand{\IhadtoSplit}{%
  This is not a good structure since it bogusly splits a long code into
  a few parts, and may global variables. Comparing to the non-split
  version, the split one is just a little better.
}
\let\tableautorefname=\tablename%
\renewcommand{\subsectionautorefname}{}
\newcommand{\refnm}[1]{%
  \autoref{#1} ``\nameref{#1}''%
}
%
% \EnableCrossrefs
\DisableCrossrefs
% \PageIndex
%\CodelineIndex
\makeatletter
\let\theCodelineNo=\reset@font
\AtEndDocument{%
  \immediate\write\@auxout{%
    \string\global\string\ft@lastsection=\thesection%
  }%
}
\makeatother
%\RecordChanges
%
\begin{document}
\MakeShortVerb{\|}
\let\PrintIndexO=\PrintIndex
\let\PrintIndex=\relax
\DocInput{\jobname.dtx}
\DocInclude{individual}
\DocInclude{sibling}
\DocInclude{gens}
\DocInclude{marriage}
\DocInclude{lib}
%\PrintChanges
% \PrintIndexO
\end{document}
%</driver>
% \fi
%
% \GetFileInfo{\jobname.sty}
% \title{{\spaceskip=1ex\textsf{\jobname} package \fileversion\\
%       {\small (\Gcid)}}}
% \author{Jiro Senju\\\texttt{\small jiro1010senju AT gmail DOT com}}
% \date{\filedate}
% \maketitle
%
% Draws a Family Tree.
%
% Defines a box describing an individual, and connects the multiple
% boxes by lines.
% The line represents the sibling, the parent-child relation ship, or
% the marriage.
%
% \begin{itemize}
% \item Excluding the marriage box, you can get a maleline\slash patrilineal
%   tree, or a femaleline\slash matrilineal tree.
% \item For Japanese, |jlreq.cls| vertical option (|tate|) is supported.
% \end{itemize}
% \bigskip
%
% \setcounter{tocdepth}{2}
% \columnseprule=\arrayrulewidth
% \begin{multicols}{2}
% \tableofcontents
% \end{multicols}
% \bigskip
%
% \setcounter{secnumdepth}{0}
% \section{Introduction}
%
% Family Tree is interesting.
% But also, it can be really complicated, especially including the
% siblings and marriages.
%
% Graphviz (|dot(1)|) is a good tool to draw a family tree, but I want
% more straightforward understandability.
% Here I try developing some macros to draw a family tree easily.
% I am not a TeXnician, but I hope it helps someone who wants to draw
% and view a large family tree.
% Tested on TeX Live 2019.
% Any comments will be appreciated.
%
% \subsection{Development}
%
% Basically all \CS s have a prefix ``|ft|''.
% But non-prefix names are also defined by |\let| as an alias\slash
% synonym, to improve the usability and the visibility.
% Obviously only when the name is undefined.
% If \CS{} name is already defined and |familytree| pkg cannot define the alias,
% |\message| is generated.
% This document tries using the |ft|-less alias name, but a few are used
% with the prefixed name.
% \smallskip
%
% There are multiple |dtx| files, but the generated |sty| is only one.
% |lib.dtx| is described at last of this document, but it comes
% first in |sty|.
% \medskip
%
% In the beginning, I was going to implement using |\hbox| and |\vbox|
% or |tabular| environment. I thought they would be enough.
% After defining the boxes, I would connect them by lines, then I got a
% trouble.
% |latex| adjusts the position of the boxes by inserting a glue or
% something, and their connection points are shifted. I could not find a
% good universal solution.
% Can TikZ or something define the absolute coordinates and the lines? I
% don't know.
% To connect the lines, I had to choose |picture| environment.
%
% The depth of a character was another trouble, or I don't have enough
% experience and
% knowledge. To layout the boxes in |picture|, I have to consider the
% depth of the box.
% To support the Japanese in vertical mode, the depth is very important. In
% horizontal mode, the depth is very alike of the English alphabets. But
% in vertical mode, the depth is a half size of a character.
% I didn't know that, and it took very long time for me.
%
% \subsection{Structure}
%
% Defines every element as a box in the tree, and connects them by lines.
% In defining a box, we also define its connection points which make the
% box to be connectable later.
%
% These are the elements.
%
% \begin{enumerate}
% \item individual box
%   \begin{itemize}
%   \item child mark to represent one is adopted or biological child
%   \item one's title
%   \item one's name
%   \item maleline\slash femaleline for the patrilineal\slash matrilineal tree
%   \item additional information\par
%     birth\slash death year-month-date, nickname, or anything
%   \end{itemize}
%
% \item sibling box
%   \begin{itemize}
%   \item a line between the child marks
%   \item interval box to make a space between individuals
%   \end{itemize}
%
% \item marriage box
%   \begin{itemize}
%   \item marriage line (double line) to connect the husband and the wife
%   \end{itemize}
%
% \item parent-child relationship or generations box
% \end{enumerate}
%
% \setcounter{secnumdepth}{3}
