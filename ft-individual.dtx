% \iffalse
% Copyright 2022 Jiro Senju
%
% This package is free software; you can redistribute it and/or modify
% it under the terms of the GNU General Public License as published by
% the Free Software Foundation; either version 2 of the License, or
% any later version.
%
% This package is distributed in the hope that it will be useful,
% but WITHOUT ANY WARRANTY; without even the implied warranty of
% MERCHANTABILITY or FITNESS FOR A PARTICULAR PURPOSE.  See the
% GNU General Public License for more details.
%
% You should have received a copy of the GNU General Public License
% along with this package.  If not, see <http://www.gnu.org/licenses/>.
% \fi
%
% \section{Individual Box (\texttt{ft-individual.dtx})}
%
% \DescribeMacro{\indvdldef}
% \indent
% \cmd{\indvdldef
%   \oarg{child mark}
%   \marg{new box name}
%   \oarg{title}
%   \marg{individual name}
%   \oarg{additional info}
%   \oarg{maleline xlength}
% }
% \medskip
%
% Defines a new box named \meta{new box name}, and layouts the
% given arguments in that box.
%
% \begin{enumerate}
% \item \oarg{child mark}
% \par
% The type of a line which is located ahead of the individual name.
% \par
% Specify one of |\ftbiological|,
% |\ftadopted|,
% |\fttop|,
% or |\ftblank|.
% Details are discussed later.
%
% \item \marg{new box name}
% \par
% The name of a new box.
% \par
% This is not a \CS{} (no backslash).
%
% \item \oarg{title}
% \par
% One's title.
% \par
% Put at the shoulder of the individual name with a smaller font
% (|\relsize{-2}|).
%
% \item \marg{individual name}
% \par
% One's name.
%
% \item \oarg{additional info}
% \par
% Additional information.
% \par
% Anything can be added such as birth\slash death year-month-date and alias.
% Put with a little smaller font (|\relsize{-1}|) at next to the individual name.
% Also the attributes (discussed later),
% |\ftmaleline| (|\ftfemaleline|, |\ftpatrilineal|, |\ftmatrilineal|),
% \allowbreak|\fthaschild|,
% |\ftprivate| are specified here.
%
% \item \oarg{maleline xlength}
% \par
% When you draw a line from an individual (instead of a marriage box) to
% a child, the length of the line can be adjusted by this argument.
% But in the sibling box, the line length will be adjusted automatically
% regardless this argument.
% \end{enumerate}
%
% Also |\indvdldef| defines some connection points as \CS. The origin is
% left-bottom of the box and the unit is |pt|.
%
% \begin{table}
% \hfil\begin{tabular}{ll}
% \meta{box name}|nameX| &
% the tail of the individual name
% \\
%
% \meta{box name}|nameCX| &
% the center of the width of individual name
% \\
%
% \meta{box name}|nameCY| &
% the center of the height of individual name
% \end{tabular}
% \end{table}
%
% For \oarg{child mark}, you can specify one of these values.
%
% \begin{itemize}
% \item |\biological| or |\ftbiological|
% \par
% a single line to represent a biological\slash real child.
%
% \item |\adopted| or |\ftadopted|
% \par
% a double line to represent an adopted child.
%
% \item |\fttop| (or |\top|)
% \par
% no line at all.
%
% \item |\blank| or |\ftblank|
% \par
% put a space whose size is equal to the |\biological| or |\adopted| mark,
% assuming used for the spouse in a marriage box.
% \end{itemize}
% \bigskip
%
% For \oarg{additional info}, you can specify any of these attributes.
%
% \begin{itemize}
% \item |\haschild| or |\fthaschild|
% \par
% represents a spouse who has a child.
% draws a line in a marriage box to their child from the center of the
% double line which represents a marriage.
%
% \item |\private| or |\ftprivate|
% \par
% represents the marriage is not official.
% draws a dashed double line.
%
% \item |\maleline| or |\ftmaleline|
% \par
% draws a line to a child from the individual name instead of the double
% line in the marriage box.
% |\ftfemaleline|,
% |\ftpatrilineal|,
% |\ftmatrilineal| are all equivalent.
% \end{itemize}
% \bigskip
%
% The individual box consists of a few smaller boxes.
%
% \begin{itemize}
% \item |\ft@titlebox|
% \item |\ft@namebox|
% \par
% In |\ft@namebox|, there are |\ft@cmarkbox| which represents the
% \meta{child mark}
% and |\ft@malelinebox| which represents \meta{maleline}.
%
% \item |\ft@optbox|
% \end{itemize}
%
% %%%%%%%%%%%%%%%%%%%%%%%%%%%%%%%%%%%%%%%
% \subsection{Customization}
%
% \DescribeMacro{\nameboxcfg}
% \indent
% \cmd{\nameboxcfg
%   \marg{space from the child mark}
%   \marg{font}
%   \marg{space to the maleline}
%   \marg{maleline length}
% }
% \medskip
%
% \DescribeMacro{\cmarkboxcfg}
% \cmd{\cmarkboxcfg
%   \marg{space between two lines, for adopted}
%   \marg{line length}
% }
% \medskip
%
% \DescribeMacro{\titleboxcfg}
% \cmd{\titleboxcfg
%   \marg{indent}
%   \marg{font}
%   \marg{linestretch}
%   \marg{vspace to the individual name}
% }
% \medskip
%
% \DescribeMacro{\optboxcfg}
% \cmd{\optboxcfg
%   \marg{vspace from the individual name}
%   \marg{indent}
%   \marg{font}
%   \marg{linestretch}
% }
%
% %%%%%%%%%%%%%%%%%%%%%%%%%%%%%%%%%%%%%%%
% \needspace{12ex}
% \subsection{Example}
%
% \begin{enumerate}
% \item
% \srcfig{fig1Robert1}
%
% \item
% \srcfig{fig1Robert2}
% \end{enumerate}
%
% %%%%%%%%%%%%%%%%%%%%%%%%%%%%%%%%%%%%%%%
% \subsectImpl
%
% %%%%%%%%%%%%%%%%%%%%%%%%%%%%%%%%%%%%%%%
% \subsubsection{Child-mark box}
%
% \parag{Customization}
%
% \DescribeMacro{\ftcmarkboxcfg}
% \DescribeMacro{\cmarkboxcfg}
%
%    \begin{macrocode}
\newlength{\ft@cmarkbox@adopted@sep}
\setlength{\ft@cmarkbox@adopted@sep}{4pt}
\newlength{\ft@cmarkbox@length}
\setlength{\ft@cmarkbox@length}{1\ft@unit}
\newcommand{\ftcmarkboxcfg}[2]{% adopted-sep line-length
  \ifx#1\empty\else%
    \setlength{\ft@cmarkbox@adopted@sep}{#1}%
  \fi%
  \ifx#2\empty\else%
    \setlength{\ft@cmarkbox@length}{#2}%
  \fi%
}
\ft@alias{cmarkboxcfg}
%    \end{macrocode}
%
% \parag{Define a box}
%
% \DescribeMacro{\fttop}
% \DescribeMacro{\top}
% \DescribeMacro{\ftbiological}
% \DescribeMacro{\biological}
% \DescribeMacro{\ftadopted}
% \DescribeMacro{\adopted}
% \DescribeMacro{\ftblank}
% \DescribeMacro{\blank}
%
%    \begin{macrocode}
\chardef\fttop=0
\chardef\ftbiological=1
\chardef\ftadopted=2
\chardef\ftblank=3
\ft@alias{top}
\ft@alias{biological}
\ft@alias{adopted}
\ft@alias{blank}
%    \end{macrocode}
%
% \DescribeMacro{\ft@indvdl@nameH}
% \DescribeMacro{\ft@indvdl@nameD}
% \DescribeMacro{\ft@indvdl@nameCY}
% \DescribeMacro{\ft@do@cmarkdef}
% \DescribeMacro{\ft@cmarkdef}
%
%    \begin{macrocode}
\newsavebox{\ft@cmarkbox}
\newlength{\ft@cmarkW}
\newlength{\ft@indvdl@nameH}
\newlength{\ft@indvdl@nameD}
\newlength{\ft@indvdl@nameCY}
\newcommand{\ft@do@cmarkdef}[4]{% cmarkbox-name cmark width line-length
  \ft@savenamebox{#1}{%
    \edef\@w{\strip@pt#3}%
    \edef\@h{\strip@pt\ft@indvdl@nameH}%
    \edef\@l{\strip@pt#4}%
    \begin{picture}(\@w,\@h)%
      \ft@dbgframe{\@w,\@h}%
      \ifcase#2% fttop
      \or% ftbiological
        \put(0,\strip@pt\ft@indvdl@nameCY){\line(1,0){\@l}}%
      \or% ftadopted
        \ft@y=\dimexpr\ft@indvdl@nameCY%
          + .5\ft@cmarkbox@adopted@sep\relax%
        \put(0,\strip@pt\ft@y){\line(1,0){\@l}}%
        \advance\ft@y -\ft@cmarkbox@adopted@sep%
        \put(0,\strip@pt\ft@y){\line(1,0){\@l}}%
      \or% \ftblank
      \fi%
    \end{picture}%
  }%
  \ft@nameboxsz{#1}{\ft@indvdl@nameH}{\ft@indvdl@nameD}%
}
%    \end{macrocode}
%
%    \begin{macrocode}
\newcommand{\ft@cmarkdef}[4]{%
  % cmarkbox-name cmark cmark-xlength cmark-space
  \ft@savenamebox{#1}{}% empty
  \ifcase#2% fttop
  \else%
    \ft@len=\dimexpr\ft@cmarkbox@length + #3\relax\relax%
    \ifdim\ft@len>0pt%
      \ft@do@cmarkdef{#1}{#2}{\dimexpr\ft@len + #4\relax}{\ft@len}%
    \fi%
  \fi%
  \ft@dbgbox{\@nameuse{#1}}%
}
%    \end{macrocode}
%
% %%%%%%%%%%%%%%%%%%%%%%%%%%%%%%%%%%%%%%%
% \needspace{3\baselineskip}
% \subsubsection{Additional info box}
% \nopagebreak
%
% \parag{Customization}
%
% \DescribeMacro{\ftoptboxcfg}
% \DescribeMacro{\optboxcfg}
%
%    \begin{macrocode}
\newlength{\ft@optbox@vsp}
\setlength{\ft@optbox@vsp}{.1\baselineskip}
\newlength{\ft@optbox@indent}
\setlength{\ft@optbox@indent}{1.1\ft@unit}
\newcommand{\ft@optbox@font}{\relsize{-1}}%{\footnotesize}
\newcommand{\ft@optbox@linestretch}{.75}
\newcommand{\ftoptboxcfg}[4]{% vsp indent font linestretch
  \ifx#1\empty\else%
    \setlength{\ft@optbox@vsp}{#1}%
  \fi%
  \ifx#2\empty\else%
    \setlength{\ft@optbox@indent}{#2}%
  \fi%
  \ifx#3\empty\else%
    \renewcommand{\ft@optbox@font}{#3}%
  \fi%
  \ifx#4\empty\else%
    \renewcommand{\ft@optbox@linestretch}{#4}%
  \fi%
}
\ft@alias{optboxcfg}
%    \end{macrocode}
%
% \parag{Parsing}
%
% \DescribeMacro{\ftmaleline}
% \DescribeMacro{\maleline}
% \DescribeMacro{\fthaschild}
% \DescribeMacro{\haschild}
% \DescribeMacro{\ftprivate}
% \DescribeMacro{\private}
%
%    \begin{macrocode}
\newcommand{\ftmaleline}{ft@attr0}
\let\ftfemaleline=\ftmaleline
\let\ftpatrilineal=\ftmaleline
\let\ftmatrilineal=\ftmaleline
\newcommand{\fthaschild}{ft@attr1}
\newcommand{\ftprivate}{ft@attr2}
\ft@alias{maleline}
\ft@alias{femaleline}
\ft@alias{patrilineal}
\ft@alias{matrilineal}
\ft@alias{haschild}
\ft@alias{private}
%    \end{macrocode}
%
% \DescribeMacro{\ft@optlist}
%
% Extracts the attributes from \meta{option-list} (which is
% \meta{additional info} itself) and sets a global flag whose name is
% generated using \meta{box-name}.
% Other than the attributes are appended another list, \meta{list} which
% will be printed later.
%
%    \begin{macrocode}
\newcommand{\ft@optlist}[3]{% list box-name option-list
  \edef\@male{\ftmaleline}%
  \edef\@hasch{\fthaschild}%
  \edef\@priv{\ftprivate}%
  \@for\@temptokena:=#3\do{%
    \edef\@opt{\@temptokena}%
    \ft@dbgmsg{\@opt}%
    \ifx\@opt\@male%
      \global\ft@malelinetrue%
      \ft@namexdef{#2hasmaleline}{\ftmaleline}%
    \else%
      \ifx\@opt\@hasch%
        \ft@namexdef{#2haschild}{\fthaschild}%
      \else%
        \ifx\@opt\@priv%
          \ft@namexdef{#2private}{\ftprivate}%
        \else%
          \ft@list@append{#1}{\@opt}%
        \fi%
      \fi%
    \fi%
  }%
}
%    \end{macrocode}
%
% \parag{Define a box}
%
% \DescribeMacro{\ft@opt}
%
%    \begin{macrocode}
\newsavebox{\ft@optbox}
\newcommand{\ft@opt}[1]{% option-list
  \savebox{\ft@optbox}{%
    \hspace{\ft@cmarkW}%
    \hspace{\ft@optbox@indent}%
    \vbox{%
      \def\baselinestretch{\ft@optbox@linestretch}%
      \ft@optbox@font%
      \vspace{\ft@optbox@vsp}%
      \@for\@temptokena:=#1\do{%
        \vss%
        \hbox{\@temptokena}%
        \setbox\@tempboxa=\hbox{\@temptokena}%
        \global\ft@depth=\dp\@tempboxa%
      }%
    }%
  }%
  \ft@len=\dimexpr\ht\ft@optbox + \dp\ft@optbox - \ft@depth\relax%
  \ft@boxsz{\ft@optbox}{\ft@len}{\ft@depth}%
  \ft@dbgbox{\ft@optbox}%
}
%    \end{macrocode}
%
% %%%%%%%%%%%%%%%%%%%%%%%%%%%%%%%%%%%%%%%
% \subsubsection{Name box}
%
% \parag{Customization}
%
% \DescribeMacro{\ftnameboxcfg}
% \DescribeMacro{\nameboxcfg}
%
%    \begin{macrocode}
\newlength{\ft@namebox@sp}
\setlength{\ft@namebox@sp}{.25\ft@unit}
\newcommand{\ft@namebox@font}{}% empty
\newlength{\ft@namebox@maleline@sp}
\setlength{\ft@namebox@maleline@sp}{.25\ft@unit}
\newlength{\ft@namebox@maleline@length}
\setlength{\ft@namebox@maleline@length}{1.5\ft@unit}
\newcommand{\ftnameboxcfg}[4]{% sp font maleline-sp maleline-length
  \ifx#1\empty\else%
    \setlength{\ft@namebox@sp}{#1}%
  \fi%
  \ifx#2\empty\else%
    \renewcommand{\ft@namebox@font}{#2}%
  \fi%
  \ifx#3\empty\else%
    \setlength{\ft@namebox@maleline@sp}{#3}%
  \fi%
  \ifx#4\empty\else%
    \setlength{\ft@namebox@maleline@length}{#4}%
  \fi%
}
\ft@alias{nameboxcfg}
%    \end{macrocode}
%
% \parag{Define a box}
%
% \DescribeMacro{\ft@name}
%
% Internally layouts |\ft@cmarkbox|, \meta{individual-name}, and
% |\ft@malelinebox|.
% It was difficult to set the length of |maleline|, ie. where to begin the
% line. Which is better for the head of |maleline|, at the box end of the
% \meta{individual-name} or at the box end including all the
% \meta{title}, \meta{individual-name} and \meta{additional info}?
%
% The latter looks good, especially when the defined individual box is
% used alone. But it makes hard for other box definitions to calculate the
% extra line length to align equal.
% For such calculation, the former is better since it just has to calculate
% the difference of the length of names.
% Finally, I decided to start the line at the end of
% \meta{individual-name}, and the default length is
% |\ft@namebox@maleline@length|.
%
% In other words, it can happen when an individual box is used alone, the
% \meta{title} or the \meta{additional info} may be longer than |maleline|.
%
%    \begin{macrocode}
\newif\ifft@maleline
\newsavebox{\ft@malelinebox}
\newsavebox{\ft@namebox}
\newcommand{\ft@name}[3]{% box-name individual-name maleline-xlength
  \setbox\@tempboxa=\hbox{\ft@namebox@font#2}%
  \ft@len=\dimexpr\wd\@tempboxa + \ft@cmarkW%
    + \ft@namebox@maleline@sp\relax%
  \ft@namexdefstrip{#1nameX}{\ft@len}%
  %
  \ft@len=\dimexpr\wd\@tempboxa/2 + \ft@cmarkW\relax%
  \ft@namexdefstrip{#1nameCX}{\ft@len}%
  %
  \def\@nm{\ft@namebox@font#2}%
  \ifft@maleline%
    % why are two 'relax'es necessary?
    \@tempskipb=\dimexpr\ft@namebox@maleline@length%
      - \ft@cmarkbox@length + #3\relax\relax%
    \ifdim\@tempskipb>0pt%
      \ft@cmarkdef{ft@malelinebox}{\ftbiological}{\@tempskipb}{0pt}%
      \def\@nm{%
        {\ft@namebox@font#2}%
        \hspace{\ft@namebox@maleline@sp}%
        \usebox{\ft@malelinebox}%
      }%
    \fi%
  \fi%
  \ft@dbgmsg{H \the\ht\@tempboxa, D \the\dp\@tempboxa}%
  \ft@dbgmsg{H \the\ht\ft@cmarkbox, D \the\dp\ft@cmarkbox}%
  \savebox{\ft@namebox}{%
    \usebox{\ft@cmarkbox}%
    \@nm%
  }%
  \ft@boxsz{\ft@namebox}{\ht\@tempboxa}{\dp\@tempboxa}%
  \ft@dbgbox{\ft@namebox}%
}
%    \end{macrocode}
%
% %%%%%%%%%%%%%%%%%%%%%%%%%%%%%%%%%%%%%%%
% \needspace{2\baselineskip}
% \subsubsection{Title box}
%
% \parag{Customization}
%
% \hyphenation{line-stretch}
% \DescribeMacro{\fttitleboxcfg}
% \DescribeMacro{\titleboxcfg}
%
%    \begin{macrocode}
\newlength{\ft@titlebox@indent}
\setlength{\ft@titlebox@indent}{-.25\ft@unit}
\newcommand{\ft@titlebox@font}{\relsize{-2}}%{\scriptsize}
\newcommand{\ft@titlebox@linestretch}{.25}
\newlength{\ft@titlebox@vsp}
\setlength{\ft@titlebox@vsp}{.1\baselineskip}
\newcommand{\fttitleboxcfg}[4]{% indent font linestretch vsp
  \ifx#1\empty\else%
    \setlength{\ft@titlebox@indent}{#1}%
  \fi%
  \ifx#2\empty\else%
    \renewcommand{\ft@titlebox@font}{#2}%
  \fi%
  \ifx#3\empty\else%
    \renewcommand{\ft@titlebox@linestretch}{#3}%
  \fi%
  \ifx#4\empty\else%
    \setlength{\ft@titlebox@vsp}{#4}%
  \fi%
}
\ft@alias{titleboxcfg}
%    \end{macrocode}
%
% \parag{Define a box}
%
% \DescribeMacro{\ft@title}
%
%    \begin{macrocode}
\newsavebox{\ft@titlebox}
\newcommand{\ft@title}[1]{% title
  \ft@len=\dimexpr\ft@cmarkW + \ft@titlebox@indent\relax%
  \savebox{\ft@titlebox}{%
    \vbox{%
      \def\baselinestretch{\ft@titlebox@linestretch}%
      \ft@titlebox@font%
      %
      \setbox\@tempboxa=\hbox{#1}%
      \ft@dbgmsg{H \the\ht\@tempboxa, D \the\dp\@tempboxa}%
      \global\ft@depth=\dp\@tempboxa%
      \hbox{\hspace{\ft@len}#1}%
      \vspace{\ft@titlebox@vsp}%
      \global\advance\ft@depth \ft@titlebox@vsp%
    }%
  }%
  \ft@dbgmsg{H \the\ht\ft@titlebox, D \the\dp\ft@titlebox}%
  \ft@len=\dimexpr\ht\ft@titlebox + \dp\ft@titlebox - \ft@depth\relax%
  \ft@boxsz{\ft@titlebox}{\ft@len}{\ft@depth}%
  \ft@dbgbox{\ft@titlebox}%
}
%    \end{macrocode}
%
% %%%%%%%%%%%%%%%%%%%%%%%%%%%%%%%%%%%%%%%
% \subsubsection{Combine the boxes --- core}
%
% \parag{Generate the boxes}
%
% \DescribeMacro{\ft@indvdlbox@gen}
% \NoDescription
%    \begin{macrocode}
\newcommand{\ft@indvdlbox@gen}[4]{%
  % box-name title individual-name maleline-xlength
  \ft@width=0pt%
  \IfValueT{#2}{%
      \ft@title{#2}%
      \ft@width=\wd\ft@titlebox%
  }%
  \global\@tempswafalse%
  \ifx\ft@indvdl@opts\empty\else%
    \global\@tempswatrue%
    \ft@opt{\ft@indvdl@opts}%
    \ifdim\ft@width<\wd\ft@optbox%
      \ft@width=\wd\ft@optbox%
    \fi%
  \fi%
  \ft@name{#1}{#3}{#4}%
}
%    \end{macrocode}
%
% \parag{Calculate the size of a name box}
%
% \DescribeMacro{\ft@indvdlbox@calc}
% \NoDescription
%    \begin{macrocode}
\newcommand{\ft@indvdlbox@calc}[2]{% box-name title
  \global\ft@width=0pt%
  \global\ft@height=0pt%
  \global\ft@depth=0pt%
  \def\ft@dpri##1{\ft@dbgmsg{##1 W \the\ft@width, H \the\ft@height,%
      D \the\ft@depth}}%
  \ft@dpri{h0}%
  \IfValueT{#2}{%
    \global\ft@width=\wd\ft@titlebox%
    \global\ft@height=\dimexpr\ht\ft@titlebox + \dp\ft@titlebox\relax%
    \ft@dpri{h1}%
  }%
  %
  \global\advance\ft@height \ht\ft@namebox%
  %
  % mark this point of the height, to invert it later
  \ft@y=\dimexpr\ft@height - \ft@indvdl@nameCY\relax%
  %
  \ifdim\ft@width<\wd\ft@namebox%
    \global\ft@width=\wd\ft@namebox%
  \fi%
  \if@tempswa%
    \global\advance\ft@height \dp\ft@namebox%
  \else%
    \global\ft@depth=\dp\ft@namebox%
  \fi%
  \ft@dpri{h2}%
  %
  \if@tempswa%
    \ifdim\ft@width<\wd\ft@optbox%
      \global\ft@width=\wd\ft@optbox%
    \fi%
    \global\advance\ft@height \ht\ft@optbox%
    \global\ft@depth=\dp\ft@optbox%
    \ft@dpri{h3}%
  \fi%
  \ft@dpri{h4}%
  %
  % invert the y
  \ft@len=\dimexpr\ft@height - \ft@y\relax%
  \ft@namexdefstrip{#1nameCY}{\ft@len}%
}
%    \end{macrocode}
%
% \parag{Layout them by \texttt{picture}}
%
% \DescribeMacro{\ft@indvdlbox@layout}
% \NoDescription
%    \begin{macrocode}
\newcommand{\ft@indvdlbox@layout}[2]{% box-name title
  \ft@newnamebox{#1}{%
    \edef\@w{\strip@pt\ft@width}%
    \edef\@h{\strip@pt\ft@height}%
    \begin{picture}(\@w,\@h)%
      \ft@dbgframe{\@w,\@h}%
      \ft@dbgplot{0,\@nameuse{#1nameCY}}%
      %
      \IfValueT{#2}{%
        \advance\ft@height -\dimexpr\ht\ft@titlebox%
          + \dp\ft@titlebox\relax%
        \put(0,\strip@pt\ft@height){\usebox{\ft@titlebox}}%
      }%
      \advance\ft@height -\ht\ft@namebox%
      \put(0,\strip@pt\ft@height){\usebox{\ft@namebox}}%
      \advance\ft@height -\dp\ft@namebox%
      \if@tempswa%
        \advance\ft@height -\ht\ft@optbox%
        \put(0,\strip@pt\ft@height){\usebox{\ft@optbox}}%
      \fi%
    \end{picture}%
  }%
  % height should hold the original value
  \ft@nameboxsz{#1}{\ft@height}{\ft@depth}%
}
%    \end{macrocode}
%
% \parag{Main function to combine the boxes}
%
% \DescribeMacro{\ft@indvdlbox}
% \NoDescription
%    \begin{macrocode}
\newcommand{\ft@indvdlbox}[4]{%
  % box-name title individual-name  maleline-xlength
  \ft@indvdlbox@gen{#1}{#2}{#3}{#4}%
  \ft@indvdlbox@calc{#1}{#2}%
  \ft@indvdlbox@layout{#1}{#2}%
  %
  \ft@dbgbox[\ft@dbgplot{0,\@nameuse{#1nameCY}}%
    \ft@dbgplot{\@nameuse{#1nameCX},\strip@pt\ft@height}%
    \ft@dbgplot{\@nameuse{#1nameX},\strip@pt\ft@height}%
  ]{\@nameuse{#1}}%
}
%    \end{macrocode}
%
% %%%%%%%%%%%%%%%%%%%%%%%%%%%%%%%%%%%%%%%
% \subsubsection{Individual box --- interface}
%
% \DescribeMacro{\ftindvdldef}
% \DescribeMacro{\indvdldef}
%
%    \begin{macrocode}
% [child-mark] box-name [title] individual-name
%       [option-list...] [maleline-xlength]
\NewDocumentCommand{\ftindvdldef}{O{\ftbiological}momoO{0pt}}{%
  \setbox\@tempboxa=\hbox{\ft@namebox@font#4}%
  \global\ft@indvdl@nameH=\ht\@tempboxa%
  \global\ft@indvdl@nameD=\dp\@tempboxa%
  \if@tate%
    \global\ft@indvdl@nameCY%
      =\dimexpr(\ft@indvdl@nameH - \ft@indvdl@nameD)/2\relax%
  \else%
    \@ifundefined{jlreqsetup}{%
      % not jlreq
      \global\ft@indvdl@nameCY=\dimexpr(\ft@indvdl@nameH%
        - \ft@indvdl@nameD) / 2\relax\relax%
    }{%
      % jlreq
      % magic number! depends on font?
      \global\ft@indvdl@nameCY=\dimexpr\ft@indvdl@nameH/2 - .66pt\relax%
    }%
  \fi%
  \ft@dbgmsg{H \the\ft@indvdl@nameH, D \the\ft@indvdl@nameD,%
    CY \the\ft@indvdl@nameCY}%
  %
  \ft@cmarkdef{ft@cmarkbox}{#1}{0pt}{\ft@namebox@sp}%
  \ifcase#1% fttop
    \ft@namexdef{#2hascmark}{#1}%
  \or% ftbiological
    \ft@namexdef{#2hascmark}{#1}%
  \or% ftadopted
    \ft@namexdef{#2hascmark}{#1}%
  \fi%
  \global\ft@cmarkW=\wd\ft@cmarkbox%
  %
  \global\ft@malelinefalse%
  \xdef\ft@indvdl@opts{}%
  \IfValueT{#5}{%
    \ft@optlist{\ft@indvdl@opts}{#2}{#5}%
    \ft@dbgmsg{opt \ft@indvdl@opts}%
  }%
  %
  \ft@indvdlbox{#2}{#3}{#4}{#6}%
}
\ft@alias{indvdldef}
%    \end{macrocode}
